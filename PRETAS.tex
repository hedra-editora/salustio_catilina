\textbf{Gaio Salústio Crispo} (86--34), nascido em Amiterno, na Sabina, foi político e historiador. Teve carreira política turbulenta, com enfrentamentos a optimates como Cícero durante seu tribunado, em 52, uma expulsão do Senado, em 50, um processo de extorsão depois de seu governo da África Nova, em 45, de que foi absolvido, e o risco de nova expulsão do Senado, de que escapou por influência de César, com quem se alinhara durante a guerra civil. Depois dos Idos de Março, retira-se da vida pública e dedica-se à escrita da história de Roma.


\textbf{A conjuração de Catilina}relata o conjunto de eventos que constituíram o malogrado plano de Lúcio
Sérgio Catilina (108--62) para se assenhorear do poder em Roma, em 63,
ano do consulado de Cícero. Catilina, de família patrícia romana,
provavelmente pretor em 68, governador da província da África no intervalo
de 67 a 66, tentara por duas vezes eleger-se cônsul, tendo sido derrotado
em ambas ocasiões (64 e 63). Depois do segundo insucesso, urde uma trama elaborada
para tomar o poder, que fracassa e termina com sua morte, na batalha dos conjurados contra as forças republicanas.


\textbf{Adriano Scatolin} é doutor em Letras Clássicas pela
Universidade de São Paulo (2009), com pós-doutorado pela Universidade Paris
\versal{IV} Sorbonne (2012--2013). Atualmente é Professor Doutor da área de Língua e Literatura Latina da Universidade de
São Paulo.

