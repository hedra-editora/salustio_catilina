


\chapterspecial{Introdução}{}{Adriano Scatolin e \mbox{Marlene L.V.~Borges}}

%\section{\textls{Conjuração de Catilina}}


A \emph{Conjuração de Catilina} relata o conjunto de eventos que constituíram o
malogrado plano de Lúcio Sérgio Catilina (108--62) para se assenhorear do poder
em Roma, no ano do consulado de Cícero, 63\footnote{Todas as datas são antes de Cristo, salvo menção em contrário. As abreviações das obras antigas seguem o padrão do \emph{Oxford Latin Dictionary} e do \emph{Greek English Lexicon}, de Liddell \& Scott.}. Catilina, de família patrícia
romana (a antiquíssima \emph{gens Sergia}), provavelmente pretor  em 68, governador da província da África como propretor no
intervalo de 67 a 66, tentara por duas vezes eleger-se cônsul, sem sucesso
(64--63). Candidato ao consulado para o ano de
63, fora derrotado por Cícero e Gaio Antônio. Ao final desse ano, apoiado pela
facção dos populares, candidata-se novamente ao mesmo cargo para o ano de 62,
mas sofre outra derrota, desta vez para Lúcio Licínio Murena e Décimo Júnio
Silano.

Nos últimos meses de 63, reunindo adeptos das mais variadas ordens sociais,
decide recorrer à revolta armada para tomar o poder. Enquanto permanece em
Roma, organizando o golpe, mantém na Etrúria um exército comandado por seu
partidário Mânlio. A intenção de Catilina seria assassinar o cônsul Cícero, incendiar a Cidade e incitar os alóbroges,
uma tribo da Gália, a uma revolta contra Roma, para que, na confusão, ficasse
mais fácil tomar o poder. A trama chega aos ouvidos de Cícero, que denuncia o
teor da conjuração ao Senado e ao povo por meio dos discursos que ficaram
conhecidos como \emph{Catilinárias}. No primeiro deles, Cícero desmascara
Catilina no Senado e exige que ele se retire de Roma. Provém desse discurso a
célebre frase que Cícero dirige ao conspirador: \emph{Quo usque tandem abutere,
Catilina, patientia nostra? }(“Até quando, afinal, Catilina, abusarás de nossa
paciência?”).  Depois desse discurso, Catilina, acuado, foge de Roma e vai
juntar-se ao exército de Mânlio na Etrúria. Enquanto isso, Cícero flagra os
passos seguintes dos conjurados que Catilina deixara em Roma (Lêntulo, Cetego e
outros), passando a ter, a partir de então, provas materiais e testemunhais da
conjura. Colocado a par da situação, o Senado determina que se executem cinco
dos conjurados que haviam sido capturados em Roma.  Um exército comandado pelo
outro cônsul, Gaio Antônio, é enviado para enfrentar Catilina.  Em fevereiro de
62, em Pistoia, Catilina e suas tropas são aniquilados pelas forças
republicanas. 

A revolta de Catilina era um tema perfeitamente adequado para Salústio redigir
sua monografia histórica. Isto porque fora contemporâneo do episódio --- embora os estudiosos acreditem que não se encontrava em Roma na época da conjuração --- e
conhecera pessoalmente os principais agentes da política naquele tempo.
Soma-se a isso o fato de que podia dispor de documentação abundante por
tratar-se de fato relativamente recente, ocorrido cerca de 20 anos antes da escrita da
obra. O tema da conjuração era também conveniente a Salústio por apresentar a
oportunidade de examinar a degeneração moral que acreditava envolver a política
e os costumes ao final da República romana. No que tange a esse aspecto,
Salústio mostra, por meio do esboço que realiza do caráter de Catilina, que
este era a figura ideal para personificar essa degeneração (5.1--8).

Salústio, ao compor a \emph{Conjuração de Catilina}, estrutura a narrativa de modo
complexo, rompendo várias vezes a ordem cronológica, intercalando digressões,
discursos e retratos ao longo de todo o relato. O esquema que se apresenta a
seguir é uma das possibilidades de interpretação do percurso narrativo que o
autor realiza\footnote{Baseamo-nos, aqui, na apresentação de \versal{SYME} (1964: 60 ss.). Outras possibilidades de divisão são apresentadas por  Vretska (1976: 20--21), McGushin (1977: 11--12), Chassignet (1999: \versal{XIV}--\versal{XV}), Ramsey (2007: 22--23) e Batstone (2010: 3--7). Embora, na numeração da tradução, adotemos a convenção do texto de base de \versal{ERNOUT} (1996), com capítulos em números romanos e seções ou parágrafos em arábicos, usamos apenas estes no esquema a seguir, nas notas à tradução e no posfácio, para maior comodidade visual.}:  

\paragraph{Parte \versal{I} --- Antes do relato da conjuração}

a)	1--4: prólogo, em que justifica seu abandono da política para escrever
história.  b)	5.1--8: esboço do caráter de Catilina.  c)	5.9--13.4:
digressão sobre o declínio da moralidade pública que teria seguido a vitória
sobre Cartago. 

\paragraph{Parte \versal{II} --- A conjuração}

a)	14--16.3: descrição do caráter dos adeptos de Catilina e ações deste
quando jovem; b)	16.4--5: concepção do plano da conjuração.  c) 	17:
reunião de Catilina com seus sequazes em junho de 64 para dar início aos preparativos 
da conjuração.  d)	18--19: retorno cronológico para narrar uma conjuração
anterior contra a República, da qual Catilina teria participado.  e)	20--31:
relato da formação da conjuração, com intercalação, no capítulo 20, do discurso
de Catilina a seus adeptos, e, no capítulo 25, do retrato de Semprônia.  e)
32:  fuga de Catilina para a Etrúria.  f)	33--36.3: preparativos da
conjuração/ Catilina e Mânlio decretados inimigos públicos pelo Senado.  g)
36.4--39.5: digressão sobre as condições políticas em Roma e possíveis causas
da conjuração.	 	

\paragraph{Parte \versal{III} --- Descoberta da conjuração}

a)	39.6--50.5: as ações da conspiração em Roma, a traição dos alóbroges, a
descoberta da conjuração.  b)	51 --- discurso de César no Senado.  c)	52 ---
discurso de Catão no Senado.  d)	53--54: retrato comparativo (síncrise) de César e Catão.
e)	55 --- execução dos conspiradores em Roma.  f)	56--58: Catilina exorta
suas tropas à luta, por meio de um discurso, depois de tomar conhecimento da
execução dos companheiros em Roma.  g)	59--61: relato da batalha final e da
morte de Catilina. 

